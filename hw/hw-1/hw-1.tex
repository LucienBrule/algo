\documentclass[11pt]{article}
\usepackage[utf8]{inputenc}
\usepackage{fancyhdr}
\usepackage{amsmath}
\usepackage{amsfonts}
\usepackage{amssymb}

% Set document information
\newcommand{\myname}{Lucien Brule}
\newcommand{\professor}{Prof. Bulent Yener}
\newcommand{\classname}{CSCI 2300: Introduction to Algorithms}
\newcommand{\assignment}{Homework 1}
\newcommand{\duedate}{\today}
\title{\classname \\
\textbf{\assignment} \\
\vspace{1em}
\large{\myname} \\
\large{\professor} \\
\large{\duedate}}
\author{}
\date{}

% Set header and footer
\pagestyle{fancy}
\fancyhf{}
\rhead{\myname}
\lhead{\classname}
\rfoot{\thepage}

\begin{document}

% Cover page
    \maketitle
    \thispagestyle{empty}
    \clearpage

% Set page numbering to start on second page
    \pagenumbering{arabic}

% Small header on each subsequent page
    \pagestyle{fancy}
    \fancyhf{}
    \rhead{\myname}
    \lhead{\classname}
    \rfoot{\thepage}

% Content starts here


    \section{Problem 1}\label{sec:problem-1}

    \paragraph{(a)$f = \theta(g)$}

    \paragraph{(b)$f = O(g)$}

    \paragraph{(c) $f = \theta(g)$}

    \paragraph{(d) $f = \theta(g)$}

    \paragraph{(e) $f = \theta(g)$}

    \paragraph{(f) $f = \theta(g)$}

    \paragraph{(g) $f = \Omega(g)$}

    \paragraph{(h) $f = \Omega(g)$}

    \paragraph{(i) $f = \Omega(g)$}

    \paragraph{(j) $f = \Omega(g)$}


    \section{Problem 2}

    \paragraph{Consider the following psuedocode which takes \break the integer $n >= n >= 0$ as input:}
    \begin{verbatim}
        def bar(n):
            print("*")
            if n == 0:
                  return
            for i in range(0,n-1):
                  bar(i)
    \end{verbatim}
    \body{
        Let T(n) be the number of times the character "*" is printed by the above code with input $n >= 0$.
        What is T(n) exactly, in terms of only n ? (ie: not values like T(n-1) or T(n-2)). Prove your answer.
    }

    \paragraph{Answer:}
    \body{
        T(n) = $n + \sum_{i=0}^{n-1} T(i)$
    }

    \paragraph{Proof:}
    \body{
        \textbf{Base case:}
        Let's first verify the base case, n = 0.

        When n = 0, the code directly prints "*", and since no recursive calls are made, T(0) = 1.

        Now, let's check the equation:

        \begin{itemize}
            \item  T(0) = $0 + \sum_{i=0}^{0-1} T(i)$
            \item T(0) = 0
        \end{itemize}


        The equation doesn't hold true for the base case.

        Let's modify the equation for T(n) considering the base case:
        T(n) = $1 + \sum_{i=0}^{n-1} T(i)$

        Now let's verify the base case again:
        T(0) = $1 + \sum_{i=0}^{0-1} T(i)$
        T(0) = 1
        The modified equation holds true for the base case.

        \textbf{Inductive step:} Let's assume the modified equation holds true for n = k, and we will show that it also holds true for n = k + 1.
        We have the following equation for n = k:
        T(k) = $1 + \sum_{i=0}^{k-1} T(i)$

        Now, let's find the equation for n = k + 1:
        T(k + 1) = $1 + \sum_{i=0}^{k} T(i)$
        We can rewrite the sum as:
        T(k + 1) = $1 + \sum_{i=0}^{k-1} T(i) + T(k)$

        From our assumption, we know that:
        T(k) = $1 + \sum_{i=0}^{k-1} T(i)$

        Substituting this into the equation for T(k + 1), we get:
        T(k + 1) = $1 + (1 + \sum_{i=0}^{k-1} T(i)) + T(k)$

        Simplifying, we get:
        T(k + 1) = $1 + \sum_{i=0}^{k} T(i)$

        Therefore, the modified equation holds true for $n = k + 1$, and by induction, it holds true for all $n >= 0$.

        Thus, the correct answer for T(n) is:
        T(n) = $1 + \sum_{i=0}^{n-1} T(i)$

    }


    \section{Problem 3}
    \textbf{Problem:} Let f(n) and g(n) be asymptotically nonnegative functions. Using the basic definition of \(\theta\)-notation, prove that \(\max(f(n), g(n)) = \theta(f(n) + g(n))\).


    \textbf{Solution:}
    To show that $\max(f(n), g(n)) = \Theta(f(n) + g(n))$, we need to prove that there exist constants $c_1, c_2 > 0$ and $n_0 \geq 0$ such that for all $n \geq n_0$:

    \[
    c_1(f(n) + g(n)) \leq \max(f(n), g(n)) \leq c_2(f(n) + g(n))
    \]

    \textbf{Lower Bound:}
    Let $c_1 = \frac{1}{2}$. Then, for any $n \geq n_0$ (with $n_0 \geq 0$), we have:

    \[
    c_1(f(n) + g(n)) = \frac{1}{2}(f(n) + g(n))
    \]

    Since $f(n)$ and $g(n)$ are asymptotically nonnegative functions, at least one of them is greater than or equal to half of their sum. Therefore, we can conclude that:

    \[
    \frac{1}{2}(f(n) + g(n)) \leq \max(f(n), g(n))
    \]

    \textbf{Upper Bound:}
    Let $c_2 = 1$. Then, for any $n \geq n_0$ (with $n_0 \geq 0$), we have:

    \[
    c_2(f(n) + g(n)) = f(n) + g(n)
    \]

    Clearly, the sum of $f(n)$ and $g(n)$ is always greater than or equal to the maximum of the two. Therefore, we can conclude that:

    \[
    \max(f(n), g(n)) \leq f(n) + g(n)
    \]

    Since we have established both the lower and upper bounds, we can conclude that:

    \[
    \max(f(n), g(n)) = \Theta(f(n) + g(n))
    \]

    \section{Problem 4}

    \textbf{Problem:}

    Is $ 2^{2n} = O(2^{n}$ ,  Why?



    \textbf{Solution:}

    This statement is false. $2^{2n}$ is not $O(2^{n})$.

    To prove this, we need to show that there do not exist constants $c > 0$ and $n_0 \geq 0$ such that for all $n \geq n_0$:

    \[ 2^{2n} \leq c \cdot 2^{n} \]

    Let's assume there exists such a constant $c > 0$. Then:

    \[ 2^{2n} \leq c \cdot 2^{n} \]

    Dividing both sides by $2^n$, we get:

    \[ 2^n \leq c \]

    However, this inequality is not true for all $n \geq n_0$, because as $n$ approaches infinity, $2^n$ will also approach infinity, which contradicts the assumption that there exists a constant $c > 0$ that satisfies this inequality.

    Therefore, the statement $2^{2n} = O(2^{n})$ is incorrect.



\end{document}

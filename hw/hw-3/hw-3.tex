\documentclass[11pt]{article}
\usepackage[utf8]{inputenc}
\usepackage{amsmath}
\usepackage{amsfonts}
\usepackage{amssymb}

\title{CSCI 2300: Introduction to Algorithms \\
\textbf{Homework 3}}
\author{Lucien Brule \\
brulel@rpi.edu}
\date{January 30, 2023}

\begin{document}

\maketitle

\section{Problem 1.1}
\paragraph*{a.}
We can show that $4^{1536} \equiv 9^{4824} \pmod{35}$ using Fermat's Little Theorem.
Fermat's Little Theorem states that for any prime $p$ and any integer $a$ such that $gcd(a,p)=1$, we have $a^{p-1} \equiv 1 \pmod{p}$.
Since $35 = 5 \cdot 7$ and both $5$ and $7$ are primes, we can use Fermat's Little Theorem on both $5$ and $7$ separately:
\begin{align*}
4^{4} &\equiv 1 \pmod 5 \
9^{6} &\equiv 1 \pmod 7 \
\end{align*}
Now we can use the property of modular exponentiation
\newline to simplify $4^{1536}$ and $9^{4824}$
\begin{align*}
4^{1536} &\equiv (4^{4})^{384} \pmod{35} \
&\equiv 1^{384} \pmod{35} \
&\equiv 1 \pmod{35} \
\end{align*}
\begin{align*}
9^{4824} &\equiv (9^{6})^{804} \pmod{35} \
&\equiv 1^{804} \pmod{35} \
&\equiv 1 \pmod{35} \
\end{align*}
Finally,
\begin{align*}
4^{1536} &\equiv 9^{4824} \
&\equiv 1 \pmod {35} \
\end{align*}
So, $4^{1536} \equiv 9^{4824} \pmod{35}$.

\pagebreak

\section*{Problem 1.2}


Is $4^{1536} \equiv 9^{4824} \pmod{35}$?

Since $35=5\cdot7$, we can use the Chinese Remainder Theorem to solve the congruence.

\begin{align*}
4^{1536} &\equiv 4^{1536}\bmod 5\cdot7 \\
&\equiv 4^{1536}\bmod 5 \cdot 4^{1536}\bmod 7 \\
&\equiv 1 \cdot 1 \\
&\equiv 1 \pmod{35} \\
\end{align*}

\begin{align*}
9^{4824} &\equiv 9^{4824}\bmod 5\cdot7 \\
&\equiv 9^{4824}\bmod 5 \cdot 9^{4824}\bmod 7 \\
&\equiv 1 \cdot 1 \\
&\equiv 1 \pmod{35} \\
\end{align*}

Therefore, $4^{1536} \equiv 9^{4824} \pmod{35}$.

The Chinese Remainder Theorem is a theorem that states that if we have a system of linear congruences, with moduli that are pairwise relatively prime, then this system has a unique solution modulo the product of the moduli.

Applying this theorem to the problem above, we have:

\begin{align*}
4^{1536} &\equiv 1 \pmod 5 \
4^{1536} &\equiv 1 \pmod 7 \
9^{4824} &\equiv 1 \pmod 5 \
9^{4824} &\equiv 1 \pmod 7
\end{align*}

Since the moduli $5$ and $7$ are relatively prime, we can find a solution for $4^{1536} \equiv 9^{4824} \pmod{35}$ using the Chinese Remainder Theorem. The unique solution is $4^{1536} \equiv 9^{4824} \equiv 1 \pmod{35}$.

\pagebreak

\section*{Problem 2}

Solve for $x^{86}$ (mod 29). 

Let's start with Fermat's Little Theorem:

\begin{equation}
x^{28} \equiv 1 (\text{mod} \ 29)
\end{equation}

Using this, we can simplify $x^{86}$:

\begin{equation}
x^{86} \equiv x^2 (\text{mod} \ 29)
\end{equation}

So, we only need to find $x^2 (\text{mod} \ 29)$.

\begin{equation}
x^2 \equiv 6 (\text{mod} \ 29)
\end{equation}

This is the same as:

\begin{equation}
x^2 \equiv 64 (\text{mod} \ 29)
\end{equation}

ergo:

\begin{equation}
x^2 - 64 \equiv (x - 8)(x + 8) \equiv 0 (\text{mod} \ 29)
\end{equation}

Thus,

\begin{equation}
x \equiv 8 (\text{mod} \ 29)
\end{equation}

\begin{equation}
x \equiv 21 (\text{mod} \ 29)
\end{equation}
\pagebreak

\section*{Problem 3}

Prove that $\gcd(F_{n+1},F_n)=1$, for $n\ge1$, where $F_n$ is the $n$-th Fibonacci element.


\textbf{Base case:}
\begin{align*}
\gcd(F_2,F_1)=\gcd(1,1)=1
\end{align*}

By induction\dots

\textbf{Assumption:}
\begin{align*}
\gcd(F_n,F_{n-1})=1\text{ for some }n\ge1
\end{align*}

\textbf{Calculation:}
\begin{align*}
\gcd(F_{n+1},F_n) &= \gcd(F_{n+1},F_n)+\gcd(F_n,F_{n-1}) \\
&= \gcd(F_n,F_{n-1})+\gcd(F_{n+1},F_{n-1}) \\
&= \gcd(F_{n+1},F_{n-1})
\end{align*}

\textbf{Conclusion:}
\begin{align*}
\gcd(F_{n+1},F_n)=1
\end{align*}


\end{document}



